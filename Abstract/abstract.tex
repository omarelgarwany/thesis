% ************************** Thesis Abstract *****************************
% Use `abstract' as an option in the document class to print only the titlepage and the abstract.
\begin{abstract}

In the past two decades, genome-wide association studies (GWAS) have identified numerous genetic variants linked to various traits and diseases, including immune-mediated diseases (IMD). However, understanding the downstream effects of these genetic variations, both at the molecular and clinical levels, has proven more challenging than expected in the post-GWAS era. To address these challenges, my thesis focuses on characterizing the effects IMD-associated loci at the transcriptomic and disease sub-phenotype levels.\\

Many disease-associated variants are found in non-coding regions of the genome, making their functional interpretation elusive. Recent large-scale functional genomic datasets like GTEx and eQTLGen have linked genetic variation to gene expression differences, but most studies have primarily focused on steady-state gene expression at the tissue level, overlooking the impact of environmental factors on gene regulation in different cell types. Additionally, aspects of transcriptomic regulation such as alternative splicing have been understudied. In the first part of the thesis, I used iPSC-derived macrophages to investigate alternative splicing patterns and identify genetic variants regulating alternative splicing in different macrophage environmental contexts. The study found widespread differential splicing between stimulated and unstimulated macrophages, with context-dependent regulation and a link between IMD risk loci and alternative splicing changes.\\

The second part of the thesis adopts a clinical perspective, focusing on perianal Crohn's disease (pCD) as a case study. A GWAS meta-analysis between CD patients with and without pCD identified a significant genetic locus in the MHC region associated with pCD, previously linked to CD susceptibility. The study also investigated sporadic perianal manifestations in the general population, finding 12 significant genetic loci associated with sporadic perianal disease. An initial assessment shows that none of these loci replicate in the pCD meta-analysis, possibly suggesting distinct mechanisms driving both types of perianal manifestations. In summary, the thesis delves into the functional and clinical aspects of IMD-associated genetic variants, emphasizing the importance of alternative splicing and exploring a high-burden clinical sub-phenotype, within both disease and sporadic contexts. This research encourages further exploration of these dimensions of IMD genetics.
\end{abstract}
